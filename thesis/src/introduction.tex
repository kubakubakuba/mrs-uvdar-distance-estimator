%!TEX root = ../main.tex

\chapter{Introduction\label{chap:introduction}}

Event based cameras, in contrast to traditional frame based cameras, do not capture still frames but rather
provide an asynchronous and independent stream of intensity changes on individual pixels.
Each pixel memorizes the last intensity value and sends an event when the intensity changes above a certain threshold.

Event cameras circumvent many common issues found in traditional frame-based cameras, such as motion blur caused
by fast-moving objects. They offer significant advantages, including a high dynamic range, low latency,
and energy efficiency.
This makes them perfect for the application of agile robotics,
where the fast response time is crucial (especially in UAV swarming situations). With their sub-millisecond response time,
event cameras can provide a significant advantage over traditional cameras in these applications.
However, they also come with some drawbacks, such as the need for a different approach to
data processing and the higher cost of the camera units themselves. \cite{gallego22event}


The event data stream is represented by tuples of $\begin{bmatrix} x & y & p & t \end{bmatrix}$, $\begin{bmatrix} x & y \end{bmatrix}$ are the pixel coordinates, $t$ is the time of
intensity change and $p$ is the polarity of the change - the increase or decrease of light intensity. Images can be
reconstructed from the event stream by integrating the events over time, doing so makes the usage of normal vision 
algorithms possible, but it also goes against the main advantage of the event cameras - the low latency.

\section{Related works}

This section should contain related state-of-the-art works and their relation to the author's work.

\section{Contributions}

This section should describe the author's contributions to the field of research.

\section{Mathematical notation}

It is a good practice to define basic mathematical notation in the introduction.
See \reftab{tab:mathematical_notation} for an example.

\begin{table*}[!h]
  \scriptsize
  \centering
  \noindent\rule{\textwidth}{0.5pt}
  \begin{tabular}{lll}
    $\mathbf{x}$, $\bm{\alpha}$ & vector, pseudo-vector, or tuple\\
    $\mathbf{\hat{x}}$, $\bm{\hat{\omega}}$& unit vector or unit pseudo-vector\\
    $\mathbf{\hat{e}}_1, \mathbf{\hat{e}}_2, \mathbf{\hat{e}}_3$ & elements of the \emph{standard basis} \\
    $\mathbf{X}, \bm{\Omega}$ & matrix \\
    $\mathbf{I}$ & identity matrix \\
    $x = \mathbf{a}^\intercal\mathbf{b}$ & inner product of $\mathbf{a}$, $\mathbf{b}$ $\in \mathbb{R}^3$\\
    $\mathbf{x} = \mathbf{a}\times\mathbf{b}$ & cross product of $\mathbf{a}$, $\mathbf{b}$ $\in \mathbb{R}^3$\\
    $\mathbf{x} = \mathbf{a}\circ\mathbf{b}$ & element-wise product of $\mathbf{a}$, $\mathbf{b}$ $\in \mathbb{R}^3$ \\
    $\mathbf{x}_{(n)}$ = $\mathbf{x}^\intercal\mathbf{\hat{e}}_n$ & $\mathrm{n}^{\mathrm{th}}$ vector element (row), $\mathbf{x}, \mathbf{e} \in \mathbb{R}^3$\\
    $\mathbf{X}_{(a,b)}$ & matrix element, (row, column)\\
    $x_{d}$ & $x_d$ is \emph{desired}, a reference\\
    $\dot{x}, \ddot{x}, \dot{\ddot{x}}$, $\ddot{\ddot{x}}$ & ${1^{\mathrm{st}}}$, ${2^{\mathrm{nd}}}$, ${3^{\mathrm{rd}}}$, and ${4^{\mathrm{th}}}$ time derivative of $x$\\
    $x_{[n]}$ & $x$ at the sample $n$ \\
    $\mathbf{A}, \mathbf{B}, \mathbf{x}$ & LTI system matrix, input matrix and input vector\\
    \emph{SO(3)} & 3D special orthogonal group of rotations\\
    \emph{SE(3)} & \emph{SO(3)}~$\times~\mathbb{R}^3$, special Euclidean group\\
  \end{tabular}
  \noindent\rule{\textwidth}{0.5pt}
  \caption{Mathematical notation, nomenclature and notable symbols.}
  \label{tab:mathematical_notation}
\end{table*}
