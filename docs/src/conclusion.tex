%!TEX root = ../main.tex

\chapter{Conclusion\label{chap:conclusion}}

This work has shown the influence of distance and frequency of modulated light sources on the response of an event based camera.
We have shown that the average number of events decreases with the increase of distance and can be approximated with
the inverse square law. The average number of events also decreases with the increase of frequency, as the camera is not
able to capture all the changes generated by the light source. What is also possible to observe from the results is the relatively
small distance range, in which the camera detects higher frequencies.

The influence of the rotation angle has also been examined, and has shown that the light source of each of the UAV arms can be
approximated with a lambertian emission model. Each of the arms can then be approximated as two lambertian sources, shifted
by $\pm 45$ degrees from the center of the arm. The real emission pattern approaches the theoretical model with the increase
of distance.

The RSSR method will be used in the subsequent bachelor thesis, where the UAV uses all of its 4 arms with different frequencies
to localize itself in the environment. This will be done by measuring the ratio of the received light intensity from each of the
arms. The camera will need to be calibrated using a calibrattion lattice of LEDs, the video data will need to be morphed
to get rid of the fish eye lens effect and to measure distances correctly.