%!TEX root = ../main.tex

\chapter{Data processing\label{chap:data_processing}}

The data has ben analyzed using Python in a Jupyter notebook\footnote{Source code is available at: \url{https://github.com/kubakubakuba/mrs-uvdar-distance-estimator}}, with libraries from Metavision SDK.

\section{Distance - frequency influence}

The distance frequency dataset has recordings of the UAV placed at 0 and 45 degrees relative to the event camera. This ensures data is captured from either
one LED source pointed directly at the event camera, or 2 LED sources pointed at the event camera at an angle. At each distance a new subdirectory was
created, with its name being the distance in meters and the content being the recordings of various frequencies in sequence, ordered by the recording timetamp.

With the range of frequencies\footnote{The frequencies represented in this list are the actual frequencies sent to the UVDAR unit. The preserved frequencies
are half of the values in this list - UVDAR recognizes the frequency with a reference to the length of the sequence (2 for on/off).} and distances being

\begin{lstlisting}
frequencies_Hz = [10, 25, 50, 100, 250, 500, 1000, 2500, 5000, 10000, 20000, 30000]
distances_m = [1.0, 1.2, 1.4, 1.6, 1.8, 2.0, 2.2, 2.4, 2.6, 2.8, 3.0, 4.0, 5.0]
\end{lstlisting}

We can load the dataset stored in raw files into a matrix representing the distances and frequencies, then load a select amount of events from each file.
The data is then resampled to a 1D array by summing polarities over a select bin width. Peaks in this signal are then analyzed by scipy's findpeaks function,
and the average amount of events with the standard deviation is calculated for each frequency and distance.

We can see the influence of distance and frequency on the average number of events on
\reffig{fig:dist} and \reffig{fig:freqs} respectively. If we fit a inverse square law function,
which can be expressed as
\begin{equation}
	\text{intensity} \propto \frac{1}{\text{distance}^2}
\end{equation}
to the data shown in \reffig{fig:fit1}, we can that the model approximates the data reasonably well.
More complex functions could be used to fit the data, but it would rather tend to overfitting instead
of representing the data in a more general way.

\begin{figure}[htbp]
	\centering
	\subfloat[Influence of distance on the average number of events.] {
	  \includegraphics[width=0.5\textwidth]{./fig/plots/dist.pdf}
	  \label{fig:dist_1}
	}
	\subfloat[Influence of distance on the log of average number of events.] {
	  \includegraphics[width=0.5\textwidth]{./fig/plots/distlog.pdf}
	  \label{fig:dist_2}
	}
	\caption{
  The influence of distance on the average number of events with the UAV rotated 0 degrees relative to the event camera on \reffig{fig:dist_1}, and with the log of the average number of events on \reffig{fig:dist_2}.
  }
	\label{fig:dist}
\end{figure}

\begin{figure}[htbp]
	\centering
	\subfloat[Influence of frequency on the average number of events.] {
	  \includegraphics[width=0.5\textwidth]{./fig/plots/freqs.pdf}
	  \label{fig:freqs_1}
	}
	\subfloat[Influence of frequency on the log of average number of events.] {
	  \includegraphics[width=0.5\textwidth]{./fig/plots/freqslog.pdf}
	  \label{fig:freqs_2}
	}
	\caption{
  The influence of frequency on the average number of events with the UAV rotated 0 degrees relative to the event camera on \reffig{fig:freqs_1}, and with the log of the average number of events on \reffig{fig:freqs_2}.
  }
	\label{fig:freqs}
\end{figure}

\begin{figure}[htbp]
	\centering
	\includegraphics[width=0.75\textwidth]{./fig/plots/fit.pdf}
	\caption{Influence of distance data fitted with polynomial and exponential function.}
	\label{fig:fit1}
  \end{figure}

\newpage

\subsection{Rotation angle influence}

From the manufacturers datasheet for the UV LEDs\footnote{The datasheet of ProLight PM2B-1LLE 1W UV Power LED can be obtained from \url{https://www.tme.eu/Document/9dfb498784ffdd07892a42f4f17c6f37/PM2B-1LLE-DTE.pdf}}
used in the UVDAR system, we can learn that the LEDs have a lambertian radiation pattern,
which can be seen on \reffig{fig:lambertian}.

\begin {figure}[htbp]
	\centering
	\includegraphics[width=0.75\textwidth]{./fig/plots/lambertian/lambertian.pdf}
	\caption{Lambertian radiation pattern of the UV LED.}
	\label{fig:lambertian}
\end{figure}

This means that the intensity of the light emmited from the LED decreases with the cosine
of the angle between the normal of the LED and the direction of the light \refeq{eq:lambertian}.

\begin{equation}
	I(\theta) = I_0\cos(\theta)
	\label{eq:lambertian}
\end{equation}

If we shift those distributions by $\pm 45$ degrees and sum them together, we can see the
theoretical distribution of the light emmited from the singular UAV arm \reffig{fig:lambert_combined}.

\begin {figure}[htbp]
	\centering
	\includegraphics[width=0.75\textwidth]{./fig/plots/lambertian/3lambertian.pdf}
	\caption{Radiation pattern of two lambertian light sources shifted by $\pm 45$ degrees.}
	\label{fig:lambert_combined}
\end{figure}